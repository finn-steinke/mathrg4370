\documentclass[12pt]{article}
\usepackage{amsmath}
\usepackage[margin=1in]{geometry}
\usepackage{hyperref}

\begin{document}

\begin{center}
{\Large \textbf{MATH*RG*4370 Geometry, Seminar Series Outline, F25/W26}} \\[1em]
Seminar Series: MATH*RG*4370, Geometry \\
Time: 12:00pm - 1:00pm \\
Room: MACN 530
\end{center}

\vspace{1em}

\noindent
\textbf{Primary Lecturer:} Finn Steinke \\
\textbf{Office:} N/A \\
\textbf{Email:} fsteinke@uoguelph.ca \\
\textbf{Office Hours:} N/A

\vspace{1em}

\noindent
\textbf{Course Website:} Structured notes will be posted on \url{github.com/finn-steinke}.

\vspace{1em}

\noindent
\textbf{Text(s):} 
\begin{itemize}
    \item John Stillwell, \emph{The Four Pillars of Geometry}, Springer, 2005. \textbf{(required, provided)}
    \item Kristopher Tapp, \emph{Differential Geometry of Curves and Surfaces}, Springer, 2016. \textbf{(required, provided)}
    \item Fergus Cooper, \emph{Introduction to Fractals and Julia Sets}. \textbf{(required, provided)}
\end{itemize}
\textbf{Note:} only \textbf{\underline{selected chapters}} (outlined below) will be covered for each text.

\vspace{1em}

\noindent
\textbf{\underline{Recommended} Prerequisites:} 
\begin{itemize}
    \item MATH*2210 (Calculus IV), or equivalent
    \item MATH*3160 (Linear Algebra II), or equivalent
    \item MATH*3200 (Real Analysis), or equivalent$^{\boldsymbol{*}}$
    \item MATH*3260 (Complex Analysis), or equivalent$^{\boldsymbol{*}}$
\end{itemize}
\noindent $^{\boldsymbol{*}}$ -- Optional for further understanding\\

\noindent \textbf{NOTE:} there are \textbf{\underline{no required}} prerequisite courses for the seminar series, it is only \textbf{\underline{recommended}} that basic concepts from each course listed above are well understood.

\vspace{1em}

\noindent
\textbf{Seminar Series Description:}  
This seminar series aims to survey selected topics in modern geometry at an advanced-undergraduate level. 
Students will be introduced to elementary aspects of Euclidean, projective, and non-Euclidean geometries, and select topics topics from elementary differential geometry. Time permitting, select topics from fractal geometry will be explored. The emphasis lies on conceptual understanding, and extension and application of concepts from MATH*2210 and MATH*3160 in a geometric setting.\\
\indent Sessions are expected to run around $1$ hour in length, with an emphasis on informal lecture and discussion following (i.e., $45$ minutes lecture / $15$ minutes discussion and exploration). This setup is \textbf{\underline{subject to change}} depending on \textbf{\underline{student feedback}}. I am more than happy to let others give some lectures/presentations as they wish as well, otherwise I am more than happy to give an informal lecture as written.

\vspace{1em}

\noindent
\textbf{Seminar Series Outline (tentative):}
\begin{enumerate}
    \item \textbf{Euclidean, Projective, and Non-Euclidean Geometry (Weeks 1-X)}
    Euclid's axioms, straightedge and compass construction, coordinates, vectors, projective geometry, spherical geometry, construction of non-Euclidean geometry. (Text: Stillwell)
    
    \item \textbf{Elementary Differential Geometry (Weeks X-Y)}  
    Parametrized curves and surfaces, tangent vectors, curvature, first fundamental form. (Text: Tapp)
    
    \item \textbf{Elementary Fractal Geometry$^{\boldsymbol{**}}$} 
    Fixed points, Hausdorff metrics, Hausdorff dimension, Julia set, Mandelbrot set. (Text: Cooper)
\end{enumerate}

\noindent $^{\boldsymbol{**}}$ -- Covered if time permits.

\vspace{1em}

\noindent
\textbf{Assignments:} N/A (Learning is purely based on attendance and participation). 

\vspace{1em}

\noindent
\textbf{Tests and Exam:} N/A (Learning is purely based on attendance and participation).  

\vspace{1em}

\noindent
\textbf{Seminar Series Mark:}  N/A
(There is no receivable grade or credit for the seminar series).

\vspace{1em}

\noindent
\textbf{Academic Integrity:}  
Using any additional learning source is \textbf{\underline{encouraged}}, whether it is a professor, student, LLM, or secondary text. If any is used, however, it must be \textbf{\underline{cited accordingly}}. If you have ideas that are not your own, please tell us where you sourced them so we can learn more -- just common sense really. Additionally, while I recommend LLM's as a \textbf{\underline{learning tool}}, I also highly recommend fact-checking them at every possible chance with \textbf{\underline{other mathematical texts}}. It is very, very easy for an LLM to be wrong -- they are not math machines, they are people-pleasing word regurgitation machines. Again, just common sense.

\vspace{1em}

\noindent
\textbf{Seminar Series Lecture Schedule (tentative):}
\begin{itemize}
    \item Week 1: Compass and straightedge construction basics (Stillwell; Ch 1.1, 1.2, 1.3)
    \item Week 2: Axiomatic Euclidean geometry (Stillwell; Ch 2.1, 2.2, 2.3, 2.4)
    \item Week 3: Coordinates and vectors (Stillwell; Ch 3.6, 4.1, 4.2, 4.3)
    \item Week 4: Inner products and matrices (Stillwell; Ch 4.4, 4.5, 4.6, 4.7)
    \item Week 5: Perspective and projection (Stillwell; Ch 5.2, 5.3, 5.4, 5.5)
    \item Week 6: Linear fractional functions and projective planes (Stillwell; Ch 5.6, 5.7, 5.8, 6.1, 6.2, 6.3)
    \item Week 7: Transformations (Stillwell; Ch 7.1, 7.2, 7.3, 7.4)
    \item Week 8: Non-Euclidean geometry (Stillwell; Ch 8.1, 8.2, 8.3, 8.4, 8.5, 8.6)
    \item Week 9: Curves (Tapp; Ch 1.1, 1.4, 1.5)
    \item Week 10: Plane and space curves (Tapp; Ch 1.6, 1.7, 1.8)
    \item Week 11: Surfaces (Tapp; 3.1, 3.2, 3.3, 3.4, 3.5)
    \item Week 12: Isometries and the first fundamental form (Tapp; 3.6, 3.7, 3.9)
    \item Week 13: TBD (time-dependent)
\end{itemize}

\end{document}

